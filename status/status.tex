\documentclass{article}

\title{Project Status: Validated Peephole Optimization for CompCert's x86 Backend}
\author{Andrew Sackville-West, Thomas M. DuBuisson}

\begin{document}
\maketitle

\section{Introduction}

Compcert's x86 back-end is known to produce inefficient code, even
when compared to the same ARM and PPC code produced from the same
source program.  Informal observation suggests there is opportunity
for performance improvement using an architecture specific peephole
optimization.

\section{Method}
In a similar vein to work by Tristian and Leroy's Translation
Validators, we have begun implementing an ML peephole optimizer, a
Gallina validator performing symbolic execution to compare pre- and
post-optimized code, a parent ``optimize'' routine to traverse the
instructions, and proofs of the validator's correctness.  Below we
address each of these components individually.

The first attempt was to add peephole optimization directly to the ASM
stage in CompCert.  After much wrestling with existing proofs, which
broke sometimes only incidentally due to the peephole optimizer, we
decided to instead add this work as a new final stage to the CompCert
pipeline.  As a result there were minor, high level, additions in
Compiler.v to both add the optimization pass and apply an (admitted)
theorem for the end-to-end proof.

\section{Symbolic-Execution Based Validator}
The validator accepts the pre- and post-optimized code as arguments.
It then symbolicly executes each of the pieces of code and compares
the result.  Our current design is to obtain three pieces of data from
symbolic execution: a mapping of locations (registers, memory) to
values (symbolic expressions), a set of all observed locations, and a
set of ``constraints'' (operations that might produce side effects not
captured by our symbolic expressions, such as dividing by zero).

As of May 24th, the symbolic execution tracks the location/value
mapping for the most common instructions and most locations (tracking
flags is a work in progress).  Additionally, the mapping has been
reworked from a function into an association list, allowing access to
the set of observed locations.

\paragraph{Shortcomings: }
Obtaining constraints is still a ways off - we are unclear exactly
what constraints exist in the x86 architecture as well as what is
necessary to make the end-to-end compiler proof feasible.  Many
instructions, especially those working on data larger or smaller than
32 bits, have no symbolic execution.  Finally, our mapping of
locations has no concept of overlapping locations (ex: {\tt movd (x) \%eax;
mov (x+4) \%ebx}), which must either be addressed or justified.

\section{Wrapping It All Together: Optimize!}
The optimize function slides a small window across a provided
instruction stream, validating each peephole-optimized window
separately.  In this manner we reduce the likely hood of being unable
to validate due to the scope containing an instruction that is not
supported by our symbolic execution engine.  On the downside, this
means we might miss optimizations or even reject optimizations that
would be valid in the larger scope.~\footnote{For example, weakening
  {\tt mul} to {\tt shl} isn't always valid due to the x86 flags, but
  it might be if the flags are adjusted by future instructions prior
  to use.}

\section{Proof of Correctness of the Validator}
The validator's proof of correctness, necessary for an end-to-end
compiler proof, is unlikely to be complete by the end of the term.  We
have several lemmas regarding register sets and instruction execution,
but those are unlikely to be useful on the validator - it appears
having a validator is an important part of proving it's operation
correct.

To aid the CompCert end-to-end proof, we have a Lemma similar to those
for other stages (where {\tt tprog} is the optimization of {\tt prog}): {\tt
  forall (beh: program\_behavior), not\_wrong beh $\rightarrow$ Asm.exec\_program
  prog beh $\rightarrow$ Asm.exec\_program tprog beh.}.

\section{ML Peephole Optimizer}
The actual peephole optimization only contains a trivial load/store
optimization.  While we are developing a framework intended to support
a wider range of optimizations, such as instruction weakening,
constant propagation, and invariant hoisting, all these optimizations
are opaque to the verifier and it's proof.  We intend to expand the
set of optimizations prior to completion.

\section{Conclusion}
While the project is progressing, it is far behind where it should be
for the full proposal to be completed by the end of term.  A
reasonable expectation might be for an expanded ML peephole
optimization routine, functional validator using symbolic execution,
but admitted proofs for the correct operation of the validator.


\end{document}
