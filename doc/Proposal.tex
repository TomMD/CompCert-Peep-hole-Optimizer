% INITIAL PROPOSAL CONTENTS

% Your initial proposal should be one page.  It should describe the
% following:
%
% o Project/paper topic.
%                        This should be described by saying what
%                        question your project is going to answer.
\documentclass{exam}

\begin{document}
\title{Deign and Development of a Verified Peep-hole Optimizer for CompCert}

\author{Thomas M. Du{B}uisson, Andrew Sackville-West}
\maketitle

\section{Introduction}
Modern compilers are so focused on performance and portability that
correctness is almost a second class citizen.  When looking only at
the GNU Compiler Collection and Low Level Virtual Machine projects,
the CSmith project~\cite{csmith} discovered over 325 bugs where the
compilers would produce incorrectly behaving programs.

Fortunately a compiler was developed with the opposite priorities; the
CompCert~\cite{compcert} compiler uses a computer assisted theorem
prover to ensure its compiler is correct.  The downside to CompCert is
performance.  Even from casual inspection of the resulting
instructions from a CompCert compilation shows there is clearly
``low-hanging fruit'' - opportunities to notably improve performance.

We propose to add a provably correct peep-hole optimizer to the last
stage of CompCert compilation, the stage in which individual assembly
instructions are selected.  For ease of development, we will target
the x86 ComCert back-end.

\section{Approach}
% o Approach.  Are you doing experimental comparison, implementation,
%              or a research paper survey?  What methodology will you
%              use?

Following the register allocation work done by Tristin and
Leroy~\cite{regalloc}, we will perform all peep-hole optimizations in
Caml.  The CompCert framework will run the untrusted optimizer on x86
assembly code to acquire a {\it possibly correct} optimization.  Our
CompCert additions will attempt to prove the optimized assembly has
the same semantic meaning as the input provided to the optimizer.  A
failure to prove the semantic correctness results in the non-optimized
assembly being used for the final compilation.

While the method of proving two assembly routines have the same
meaning is one of the deliverables, our first area of exploration will
be reverse information flow in a manner identical to the register
allocation work.  FIXME what else?


\section{Deliverables}
% o Deliverables.  What do you plan to hand in?  Just a paper?
%                  Experimental results? Code?
There are three main deliverables: amethod of proving two assembly routines
have the same semantic meaning, an implementation of the proof in the CompCert
framework, and a peep-hole optimizer in Caml.  The fact that the actual transformation
is an optimizer (vs a obfuscation, or a resturcturing a la Native Client) is expected
to be immaterial to the proof method; as a result, the optimizer will consist of
relatively few, trivial transformations and only expanded given sufficient time.

\end{document}
